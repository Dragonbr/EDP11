% ------------------------------------------------------------------------------
% Este fichero es parte de la plantilla LaTeX para la realización de Proyectos
% Final de Grado, protegido bajo los términos de la licencia GFDL.
% Para más información, la licencia completa viene incluida en el
% fichero fdl-1.3.tex

% Copyright (C) 2012 SPI-FM. Universidad de Cádiz
% ------------------------------------------------------------------------------

\section{Objetivos}

Previa realización del proyecto se propusieron los objetivos de aprender sobre todo, no sólo en conocimientos, sino a nivel personal, aprender a afrontar y solucionar retos mayores, demostrando la capacidad de ser autosuficientes y valernos por si mismos.\\\\
Dada la oportunidad del proyecto, uno de los objetivos principales fue también ayudar, aplicar los conocimiento y capacidades para facilitar el trabajo de otro a través de la informática. Cuestión principal de estudiar esta carrera.

\section{Lecciones aprendidas}
Pese a que durante la titulación se han adquirido multidud de conocimientos relevantes de la informática, no se ha puesto a prueba la situación de valernos por nosotros mismos ante proyectos completos y reales.\\\\
Es por ello que para el proyecto se ha empleado el lenguaje de programación \textit{Python}, la biblioteca gráfica \textit{Qt}, la biblioteca \textit{Relatorio}, \textit{Git} para el repositorio, así como el IDE generador de interfaces gráficas \textit{Qt4 Designer}, nada de esto visto con anterioridad, ni en la titulación.\\\\
Otras herramientas como \textit{LATEX}, para la generación de la memoria y los manuales; \textit{DIA} para la realización de todos los diagramas de la parte de análisis y diseño de la memoria, así como el programa de edición de imagenes \textit{Gimp}, con el que se editaron las capturas de pantalla contenidas en la memoria.

\section{Trabajo futuro}
En un futuro, la aplicación se pretende ampliar, añadiendo estadísticas de seguimiento al paciente, paquetes de idiomas, un sistema de colores en base a las preferencias, así como una posterior integración con sistemas móviles y web.\\

