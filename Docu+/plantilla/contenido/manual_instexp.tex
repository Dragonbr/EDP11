% ------------------------------------------------------------------------------
% Este fichero es parte de la plantilla LaTeX para la realización de Proyectos
% Final de Grado, protegido bajo los términos de la licencia GFDL.
% Para más información, la licencia completa viene incluida en el
% fichero fdl-1.3.tex

% Copyright (C) 2012 SPI-FM. Universidad de Cádiz
% ------------------------------------------------------------------------------

Las instrucciones de instalación se detallan a continuación.

\section{Requisitos previos}
Para poder ejecutar correctamente la aplicación es necesario tener instalado ciertos paquetes, los cuales se detallan a continuación:
\begin{itemize}
\item \textit{Python}. Versiones superiores a 2.7. Normalmente bajo distribuciones Linux ya se encontrará instalado, en específico el paquete python. En cualquier caso, esta disponible en su página oficial.\\
\url{http://www.python.org/download/}

\item \textit{PyQt4}. En especial python-qt-dev, python-qt4-dev, python-qt4 bajo versiones Linux. Disponible también en su página oficial.\\
\url{http://www.riverbankcomputing.co.uk/software/pyqt/download}

\item \textit{OpenOffice} o \textit{LibreOffice}. Se encontrará instalado bajo distribuciones Linux. Disponible en sus respectivas páginas oficiales.\\
\url{http://www.openoffice.org/es/descargar/index.html}\\
\url{http://es.libreoffice.org/descarga/}

\item \textit{Relatorio} y \textit{Yaml}. Bajo distribuciones Linux se necesita el paquete python-relatorio y python-yaml. Disponible desde sus respectivas páginas oficiales.\\
\url{http://relatorio.openhex.org/}\\
\url{http://www.yaml.org/}

\item \textit{Poppler}. En las últimas versiones de distribuciones Linux, libpoppler-qt4-dev, poppler-utils, libpoppler-dev, python-poppler-qt4, se incluyen en los repositorios. Disponible desde su página oficial.\\
\url{http://poppler.freedesktop.org/}

\item \textit{Pycha}. Necesario el paquete python-pycha disponible en los repositorios. Disponible también en la página del proyecto.\\
\url{https://bitbucket.org/lgs/pycha/downloads}

\end{itemize}

\section{Procedimientos de instalación}
La aplicación es distribuida en un paquete ``.deb'', así como los propios fuentes. El paquete ``.deb'', la instalación se llevará a cabo mediante el gestor de paquetes correspondiente.\\
En el caso de los propios fuentes no hará falta instalador.

\section{Puesta en funcionamiento}
Para comenzar a utilizar la aplicación será necesario que se asegure de la correcta instalación de todos los paquetes necesarios. En instalaciones mediante deb, bastará con ejecutar el launcher. 
En el caso de los propios fuentes, se encontrará el archivo \textit{Estudio.sh} en el directorio ``edp11'', el cual ejecutará la aplicación.
