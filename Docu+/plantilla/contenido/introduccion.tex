% ------------------------------------------------------------------------------
% Este fichero es parte de la plantilla LaTeX para la realización de Proyectos
% Final de Grado, protegido bajo los términos de la licencia GFDL.
% Para más información, la licencia completa viene incluida en el
% fichero fdl-1.3.tex

% Copyright (C) 2012 SPI-FM. Universidad de Cádiz
% ------------------------------------------------------------------------------

Este Proyecto Fin de Carrera tiene como objetivos: aplicar los conocimientos adquiridos durante la titulación Ingeniería Técnica en Informática de Sistemas, aprender y utilizar nuevas herramientas que enriquecen los conocimientos que no se han podido adquirir durante la titulación y aportar una contribución al mundo del Software Libre y al sector de la nutrición a la que esta orientado dicho proyecto.

\section{Motivación}
En el mundo de hoy, existen muchos problemas de salud, cada día siendo más comunes los problemas relacionados con la comida, como son la obesidad o el colesterol. Es por ello que se hace más necesario personas que nos orienten en cuanto a alimentación nos referimos, tales como los nutricionistas.\\
Por este motivo es por el que se desarrolla dicho proyecto, para hacer más fácil la tarea de los nutricionistas, informatizando sus sistemas y ahorrándoles tiempo y esfuerzo.

\section{Objetivos y alcance del proyecto} 
Este proyecto consiste en la creación de un software que permita gestionar los hábitos alimenticios de los pacientes, así como la posibilidad de gestionar una pequeña o mediana empresa de nutricionistas, como son las franquicias de nutrición más conocidas, albergando el numeroso personal de nutricionistas que tienen al frente.\\

Como objetivos se plantean:\\
\begin{itemize}
\item Gestión de Dietistas
\item Gestión de Pacientes
\item Gestión de Recetas
\item Gestión de Ingredientes
\item Gestión de Semanarios
\end{itemize}


\section{Organización del documento}
Este documento se compone de:
\begin{itemize}
\item \textbf{Introducción}: pequeña descripción del proyecto, incluyendo objetivos, alcance y estructura.
\item \textbf{Planificación}: planificación temporal para el desarrollo del proyecto, reflejándola mediante un diagrama de "Gantt".
\item \textbf{Análisis}: análisis del sistema. Definiéndose los requisitos funcionales, diagramas de casos de uso, diagramas de secuencia y contrato de operaciones.
\item \textbf{Diseño}: diseño del sistema. Obteniéndose la base de datos, diagramas de secuencia y de clases aplicadas al diseño.
\item \textbf{Implementación}: se detallarán algunos aspectos relevantes para la implementación así como la superación de posibles problemas que se pudieran encontrar durante la implementación.
\item \textbf{Pruebas y Validaciones}: pruebas realizadas para la corroboración del funcionamiento del software.
\item \textbf{Manual de Usuario}: manual para el uso de la aplicación.
\item \textbf{Manual de instalación y explotación}: manual para la instalación de la aplicación. 
\item \textbf{Conclusiones}: valoración personal de la realización del proyecto y futuras mejoras.
\item \textbf{Bibliografía}: referencias, enlaces, libros y ayudas consultadas durante el desarrollo del proyecto.
\item \textbf{Información sobre Licencia}: licencia en la que se basa el proyecto, en concreto, GPL 3.0 y FDL 1.3.
\end{itemize}




